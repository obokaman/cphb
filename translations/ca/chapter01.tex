\chapter{Introducció}

La programació competitiva combina dos temes:
(1) el disseny d'algorismes i (2) la implementació d'algorismes.

El \key{disseny d'algorismes} consisteix en la resolució de problemes
i el pensament matemàtic.
Fan falta habilitats per analitzar problemes i resoldre'ls
creativament.
Un algorisme per resoldre un problema
ha de ser correcte i eficient,
i la dificultat consisteix sovint
sobre inventar un algorisme eficient.

Els coneixements teòrics d'algorismes
s\'on important per als programadors competitius.
Normalment, una solució a un problema és
una combinació de tècniques conegudes i
noves idees.
Les tècniques que apareixen en la programació competitiva
també constitueixen la base per a la investigació científica
d'algorismes.

La \key{implementació d'algorismes} requereix bones
habilitats de programació.
En la programació competitiva, les solucions
es classifiquen provant un algorisme implementat
fent servir un conjunt de casos de prova.
Per tant, no n'hi ha prou que la idea del
l'algoritme sigui correcte, la implementació també
ha de ser correcte.

L'estil correcte de programaci\'o en els concursos és
ser directe i concís.
Els programes s'han d'escriure ràpidament,
perquè no hi ha gaire temps disponible.
A diferència de l'enginyeria de software tradicional,
els programes són curts (normalment com a molt uns pocs
centenars de línies de codi) i no cal mantenir el
codi un cop acabat el concurs.

\section{Llenguatges de programació}

\index{llenguatge de programacio}

En l'actualitat, els llenguatges de programaci\'o que
es fan servir més als concursos s\'on
el C++, el Python i el Java.
Per exemple, a Google Code Jam 2017,
entre els 3.000 millors participants,
el 79 \% va fer servir C++,
el 16 \% Python i
el 8 \% Java \cite{goo17}.
Alguns participants també utilitzaven varis llenguatges.

Molta gent pensa que el C++ és la millor opció
per a un programador competitiu,
i el C++ gairebé sempre està disponible en els
sistemes de concurs.
Els avantatges d'utilitzar C++ són que
és un llenguatge de programaci\'o molt eficient i
que la seva biblioteca estàndard conté un
gran col·lecció d'estructures de dades i algorismes.

D'altra banda, és bo
dominar diversos llenguatges de programaci\'o i comprendre
els seus punts forts.
Per exemple, si es necessiten nombres enters grans
en el problema,
Python pot ser una bona opció, perquè aquest
llenguatge
conté operacions integrades per
càlculs amb nombres enters grans.
Tot i així, la majoria dels problemes en els concursos de programació
s'intenten preparar per a que
fer servir un llenguatge de programació específic
no sigui un avantatge injust.

Tots els programes d'exemple d'aquest llibre estan escrits en C++,
i sovint es fa servir les estructures de dades i algorismes de
la llibreria estàndard.
Els programes segueixen l'estàndard C++11,
que es pot fer servir en la majoria dels concursos actuals.
Si encara no podeu programar en C++,
ara és un bon moment per començar a aprendre.

\subsubsection{plantilla de codi C++}

Aquesta \'es una plantilla de codi C++ típica per a la
programació competitiva:

\begin{lstlisting}
#include <bits/stdc++.h>

using namespace std;

int main() {
  // Aqu\'i va el codi.
}
\end{lstlisting}

La línia \texttt{\#include} al principi
del codi és una característica del compilador \texttt{g++}
que ens permet incloure tota la biblioteca estàndard.
Per tant, no cal incloure per separat
biblioteques com ara \texttt{iostream},
\texttt{vector} i \texttt{algorithm},
sinó que estan disponibles automàticament.

La línia \texttt{using} declara
que les classes i funcions
de la biblioteca estàndard es poden fer servir directament
al codi.
Sense la línia \texttt{using} tindríem
que escriure, per exemple, \texttt{std::cout},
però ara n'hi ha prou amb escriure \texttt{cout}.

El codi es pot compilar mitjançant l'ordre següent:

\begin{lstlisting}
g++ -std=c++11 -O2 -Wall test.cpp -o test
\end{lstlisting}

Aquesta ordre produeix un fitxer binari \texttt{test}
a partir del codi font \texttt{test.cpp}.
El compilador segueix l'estàndard C++11
(\texttt{-std=c++11}),
optimitza el codi (\texttt{-O2})
i mostra avisos sobre possibles errors (\texttt{-Wall}).

\section{Entrada i sortida}

\index{entrada i sortida}

A la majoria de concursos es fan servir els canals estàndard
per a la lectura d'entrada i l'escriptura de sortida.
En C++, els canals estàndard són
\texttt{cin} per a l'entrada i \texttt{cout} per a la sortida.
Tamb\'e es poden fer servir les funcions de C
\texttt{scanf} i \texttt{printf}.

L'entrada per al programa normalment consisteix en
nombres i cadenes de text que es separen amb
espais i salts de l\'inia.
Es poden llegir des del canal \texttt{cin}
com segueix:

\begin{lstlisting}
int a, b;
string x;
cin >> a >> b >> x;
\end{lstlisting}

Aquest tipus de codi sempre funciona,
suposant que hi ha almenys un espai
o salt de línia entre cada element de l'entrada.
Per exemple, el codi anterior pot llegir les dues entrades
següents:
\begin{lstlisting}
123 456 monkey
\end{lstlisting}
\begin{lstlisting}
123 456
monkey
\end{lstlisting}
El canal \texttt{cout} es fa servir per a la sortida:
\begin{lstlisting}
int a = 123, b = 456;
string x = "monkey";
cout << a << " " << b << " " << x << "\n";
\end{lstlisting}

L'entrada i la sortida són de vegades
un coll d'ampolla en el programa.
Les línies següents al principi del codi
fan que l'entrada i la sortida siguin
més eficients:

\begin{lstlisting}
ios::sync_with_stdio(0);
cin.tie(0);
\end{lstlisting}

Tingueu en compte que el salt de línia \texttt{"\textbackslash n"}
funciona més ràpid que \texttt{endl},
perquè \texttt{endl} sempre provoca
una operació de buidat del buffer.

Les funcions C \texttt{scanf}
i \texttt{printf} són una alternativa
als canals estàndard de C++.
Normalment són una mica més ràpides,
però també són més difícils d'utilitzar.
El codi següent llegeix dos nombres enters de l'entrada:
\begin{lstlisting}
int a, b;
scanf("%d %d", &a, &b);
\end{lstlisting}
El codi següent imprimeix dos nombres enters:
\begin{lstlisting}
int a = 123, b = 456;
printf("%d %d\n", a, b);
\end{lstlisting}

De vegades, el programa hauria de llegir una línia sencera
de l'entrada, possiblement amb espais.
Això es pot aconseguir mitjançant l'ús de
la funci\'o \texttt{getline}:

\begin{lstlisting}
string s;
getline(cin, s);
\end{lstlisting}

Si es desconeix la quantitat de dades, podem fer servir el
següent bucle:
\begin{lstlisting}
while (cin >> x) {
    // codi
}
\end{lstlisting}
Aquest bucle llegeix elements de l'entrada
un rere l'altre, fins que no n'hi ha
hagi més dades disponibles.

En alguns sistemes de concurs es fa servir fitxers per
a l'entrada i la sortida.
Una solució fàcil per a això és escriure
el codi com de costum utilitzant canals estàndard,
però afegint les línies següents al començament del codi:
\begin{lstlisting}
freopen("entrada.txt", "r", stdin);
freopen("output.txt", "w", stdout);
\end{lstlisting}
Després d'això, el programa llegeix l'entrada del fitxer
``input.txt'' i escriu la sortida al fitxer
``output.txt''.

\section{Treballar amb nombres}

\index{enter}

\subsubsection{Nombres enters}

El tipus enter més utilitzat en la programació competitiva
és el tipus \texttt{int}, que és un tipus de 32 bits amb
un rang de valors de $-2^{31} \ldots 2^{31}-1$
o aproximadament $-2 \cdot 10^9 \ldots 2 \cdot 10^9$.
Si el tipus \texttt{int} no és suficient,
es pot fer servir el tipus de 64 bits \texttt{long long}.
Té un rang de valors de $-2^{63} \ldots 2^{63}-1$
o aproximadament $-9 \cdot 10^{18} \ldots 9 \cdot 10^{18}$.

El codi següent defineix una
variable de tipus \texttt{long long}:
\begin{lstlisting}
long long x = 123456789123456789LL;
\end{lstlisting}
El sufix \texttt{LL} significa que el
el tipus del nombre és \texttt{long long}.

Un error comú quan es fa servir el tipus \texttt{long long}
és fer servir algun tipus \texttt{int} en algun lloc
al codi.
Per exemple, el codi següent conté
un error subtil:

\begin{lstlisting}
int a = 123456789;
long long b = a*a;
cout << b << "\n"; // -1757895751
\end{lstlisting}

Tot i que la variable \texttt{b} és del tipus \texttt{long long},
els dos nombres de l'expressió \texttt{a*a}
són del tipus \texttt{int} i el resultat és
també del tipus \texttt{int}.
Per això, la variable \texttt{b} tindrà
un resultat incorrecte.
El problema es pot resoldre fent que el tipus
de \texttt{a} sigui \texttt{long long} o
canviant l'expressi\'o a \texttt{(long long)a*a}.

Normalment els problemes de concurs es plantegen de manera que
amb el tipus \texttt{long long} n'hi ha prou.
Tot i així, és bo saber que el compilador \texttt{g++} també proporciona
un tipus de 128 bits \texttt{\_\_int128\_t}
amb un rang de valors de
$-2^{127} \ldots 2^{127}-1$ o aproximadament $-10^{38} \ldots 10^{38}$.
Tanmateix, aquest tipus no està disponible en tots els sistemes de concurs.

\subsubsection{Aritmètica modular}

\index{residu}
\index{aritmètica modular}

Denotem per $x \bmod m$ el residu
de dividir $x$ per $m$.
Per exemple, $17 \bmod 5 = 2$,
perquè $17 = 3 \cdot 5 + 2$.

De vegades, la resposta a un problema és a
nombre molt gran però es demana que s'escrigui
la soluci\'o ``mòdul $m$'', és a dir,
el residu quan es divideix la resposta per $m$
(per exemple, ``mòdul $10^9+7$'').
D'aquesta manera, encara que la resposta real
sigui molt gran,
n'hi ha prou amb utilitzar els tipus
\texttt{int} i \texttt{long long}.

Una propietat important del residu és que
en la suma, la resta i la multiplicació,
el residu es pot prendre abans de l'operació:

\[
\begin{matrix}{rcr}
(a+b) \bmod m & = & (a \bmod m + b \bmod m) \bmod m \\
(a-b) \bmod m & = & (a \bmod m - b \bmod m) \bmod m \\
(a \cdot b) \bmod m & = & (a \bmod m \cdot b \bmod m) \bmod m
\end{matrix}
\]

Així, podem agafar el residu després de cada operació
i les xifres mai seran massa grans.

Per exemple, el codi següent calcula $n!$,
el factorial de $n$, mòdul $m$:
\begin{lstlisting}
long long x = 1;
per (int i = 2; i <= n; i++) {
    x = (x*i)%m;
}
cout << x%m << "\n";
\end{lstlisting}

Normalment volem que el residu estigui sempre
en $0\ldots m-1$.
Tanmateix, en el C++ i altres llenguatges,
el residu d'un nombre negatiu
és zero o negatiu.
Una manera fàcil d'assegurar-s'hi que no tenim
residus negatius es cal calcular primer
el residu com de costum i després afegeix $m$
si el resultat és negatiu:
\begin{lstlisting}
x = x%m;
if (x < 0) x += m;
\end{lstlisting}
Això només és necessari quan hi ha
restes en el codi i el
el residu pot arribar a ser negatiu.

\subsubsection{Nombres de coma flotant}

\index{nombre de coma flotant}

Els tipus de coma flotant en la
programació competitiva són
el \texttt{double} de 64 bits
i, com a extensió al compilador \texttt{g++},
el \texttt{long double} de 80 bits.
En la majoria dels casos, \texttt{double} és suficient,
però \texttt{long double} és més precís.

La precisió requerida de la resposta
normalment es dóna a l'enunciat del problema.
Una manera fàcil d'emetre la resposta és fer servir
la funció \texttt{printf}
i donar el nombre de decimals
a la cadena de text que especifica el format.
Per exemple, el codi següent escriu
el valor de $x$ amb 9 decimals:

\begin{lstlisting}
printf("%.9f\n", x);
\end{lstlisting}

Una dificultat quan s'utilitzen nombres de coma flotant
és que alguns nombres no es poden representar
amb precisió com a nombres de coma flotant,
i hi haurà errors d'arrodoniment.
Per exemple, el resultat del codi següent
és sorprenent:

\begin{lstlisting}
double x = 0.3*3+0.1;
printf("%.20f\n", x); // 0.9999999999999988898
\end{lstlisting}

El valor de \texttt{x} és una mica més petit que 1 degut
als errors d'arrodoniment, quan el valor correcte seria 1.

És arriscat comparar nombres de coma flotant
amb l'operador \texttt{==},
perquè és possible que, encara els valors haurien de
ser iguals, en realitat no ho s\'on degut
a errors de precisió.
Una millor manera de comparar nombres de coma flotant
és suposar que dos nombres són iguals
si la diferència entre ells és menor que $\varepsilon$,
on $\varepsilon$ és un nombre petit.

A la pràctica, els nombres es poden comparar
de la següent manera ($\varepsilon=10^{-9}$):

\begin{lstlisting}
if (abs(a-b) < 1e-9) {
    // a i b son iguals
}
\end{lstlisting}

Tingueu en compte que, tot i que els nombres de coma flotant són inexactes,
els enters fins a un cert límit encara es poden
representar amb precisió.
Per exemple, fent servir \texttt{double},
és possible representar amb precisió
nombres enters el valor absolut dels quals és com a màxim $2^{53}$.

\section{Escurçar el codi}

El codi curt és ideal en programació competitiva,
perquè els programes s'han d'escriure
el més ràpid possible.
Per això, els programadors competitius sovint defineixen
noms més curts per a tipus de dades i altres parts del codi.

\subsubsection{Noms dels tipus}
\index{tuppdef@\texttt{typedef}}
Fent servir l'ordre \texttt{typedef}
és possible donar un nom més curt
a un tipus de dades.
Per exemple, com que el nom \texttt{long long} és llarg,
podem definir un nom més curt \texttt{ll} d'aquesta manera:
\begin{lstlisting}
typedef long long ll;
\end{lstlisting}
Després d'això, el codi
\begin{lstlisting}
long long a = 123456789;
long long b = 987654321;
cout << a*b << "\n";
\end{lstlisting}
es pot escurçar de la següent manera:
\begin{lstlisting}
ll a = 123456789;
ll b = 987654321;
cout << a*b << "\n";
\end{lstlisting}

L'ordre \texttt{typedef}
també es pot fer servir amb tipus més complexos.
Per exemple, el codi següent dóna
el nom \texttt{vi} per a un vector de nombres enters
i el nom \texttt{pi} per a una parella
que conté dos nombres enters.
\begin{lstlisting}
typedef vector<int> vi;
typedef pair<int,int> pi;
\end{lstlisting}

\subsubsection{Macros}
\index{macro}
Una altra manera d'escurçar el codi és definir
\key{macros}.
Una macro significa que hi ha determinades cadenes
de text que el codi es canviarà abans de la compilació.
En C++, les macros es defineixen mitjançant la paraula
clau \texttt{\#define}.

Per exemple, podem definir les macros següents:
\begin{lstlisting}
#define F primer
#define S segon
#define PB push_back
#define MP make_pair
\end{lstlisting}
Després d'això, el codi
\begin{lstlisting}
v.push_back(make_pair(y1,x1));
v.push_back(make_pair(y2,x2));
int d = v[i].primer+v[i].segon;
\end{lstlisting}
esdev\'e:
\begin{lstlisting}
v.PB(MP(y1,x1));
v.PB(MP(y2,x2));
int d = v[i].F+v[i].S;
\end{lstlisting}

Una macro també pot tenir paràmetres,
la qual cosa fa possible escurçar bucles i altres
estructures.
Per exemple, podem definir la macro següent:
\begin{lstlisting}
#definir REP(i,a,b) for (int i = a; i <= b; i++)
\end{lstlisting}
Després d'això, el codi
\begin{lstlisting}
for (int i = 1; i <= n; i++) {
    cerca(i);
}
\end{lstlisting}
esdev\'e:
\begin{lstlisting}
REP(i,1,n) {
    cerca(i);
}
\end{lstlisting}

De vegades, les macros provoquen errors que poden ser difícils
per detectar. Per exemple, considereu la macro següent
que calcula el quadrat d'un nombre:
\begin{lstlisting}
#define SQ(a) a*a
\end{lstlisting}
Aquesta macro \emph{no} sempre funciona com s'esperava.
Per exemple, el codi
\begin{lstlisting}
cout << SQ(3+3) << "\n";
\end{lstlisting}
correspon al codi
\begin{lstlisting}
cout << 3+3*3+3 << "\n"; // 15
\end{lstlisting}

Una versió millor de la macro és la següent:
\begin{lstlisting}
#definir SQ(a) (a)*(a)
\end{lstlisting}
Ara el codi
\begin{lstlisting}
cout << SQ(3+3) << "\n";
\end{lstlisting}
correspon al codi
\begin{lstlisting}
cout << (3+3)*(3+3) << "\n"; // 36
\end{lstlisting}


\section{Matemàtiques}

Les matemàtiques tenen un paper important en la competició
programació, i no és possible arribar a ser
un programador competitiu exitós sense
tenir bones habilitats matemàtiques.
En aquesta secció es discuteixen alguns conceptes
i fórmules matemàtiques que són necess\`aries
més endavant al llibre.

\subsubsection{Fórmules de suma}

Cada suma de la forma
\[\sum_{x=1}^n x^k = 1^k+2^k+3^k+\ldots+n^k,\]
on $k$ és un nombre enter positiu,
té una fórmula tancada que és un
polinomi de grau $k+1$.
Per exemple\footnote{\index{fórmula de Faulhaber}
Fins i tot hi ha una fórmula general per a aquestes sumes,
anomenada \key{fórmula de Faulhaber},
però és massa complexa per a presentar-la aquí.},
\[\sum_{x=1}^n x = 1+2+3+\ldots+n = \frac{n(n+1)}{2}\]
i
\[\sum_{x=1}^nx^2 = 1^2+2^2+3^2+\ldots+n^2 = \frac{n(n+1)(2n+1)}{ }.\]

Una \key{progressió aritmètica} és una \index{progressió aritmètica}
seqüència de nombres
on la diferència entre dos nombres consecutius 
és una constant.
Per exemple,
\[3, 7, 11, 15\]
és una progressió aritmètica amb constant 4.
Es pot calcular la suma d'una progressió aritmètica
fent servir la fórmula
\[\underbrace{a + \cdots + b}_{n \,\, \textrm{nombres}} = \frac{n(a+b)}{2}\]
on $a$ és el primer nombre,
$b$ és l'últim nombre i
$n$ és la quantitat de nombres.
Per exemple,
\[3+7+11+15=\frac{4 \cdot (3+15)}{2} = 36.\]
La fórmula es basa en el fet
que la suma consta de $n$ nombres i
el valor de cada nombre és $(a+b)/2$ de mitjana.

\index{progressió geomètrica}
Una \key{progressió geomètrica} és una seqüència
de nombres
on el rati entre dos consecutius qualsevol
nombres és una constant.
Per exemple,
\[3,6,12,24\]
és una progressió geomètrica amb constant 2.
Es pot calcular la suma d'una progressió geomètrica
fent servir la fórmula
\[a + ak + ak^2 + \cdots + b = \frac{bk-a}{k-1}\]
on $a$ és el primer nombre,
$b$ és l'últim nombre i el
la rati entre nombres consecutius és $k$.
Per exemple,
\[3+6+12+24=\frac{24 \cdot 2 - 3}{2-1} = 45.\]

Aquesta fórmula es pot derivar de la següent manera. Sigui
\[ S = a + ak + ak^2 + \cdots + b .\]
Quan multipliquem els dos costats per $k$, obtenim
\[ kS = ak + ak^2 + ak^3 + \cdots + bk,\]
i resolent l'equació
\[ kS-S = bk-a\]
ens dóna la fórmula.

Un cas especial de la suma d'una progressió geomètrica és la fórmula
\[1+2+4+8+\ldots+2^{n-1}=2^n-1.\]

\index{suma harmònica}

Una \key{suma harmònica} és una suma de la forma
\[ \sum_{x=1}^n \frac{1}{x} = 1+\frac{1}{2}+\frac{1}{3}+\ldots+\frac{1}{n} .\]

Un límit superior per a una suma harmònica és $\log_2(n)+1$.
En concret, podem
modificar cada terme $1/k$ de manera $k$ es converteixi en
la potència més propera de dos que no superi $k$.
Per exemple, quan $n=6$, podem estimar
la suma de la següent manera:
\[ 1+\frac{1}{2}+\frac{1}{3}+\frac{1}{4}+\frac{1}{5}+\frac{1}{6} \le
1+\frac{1}{2}+\frac{1}{2}+\frac{1}{4}+\frac{1}{4}+\frac{1}{4}.\]
Aquest límit superior consta de $\log_2(n)+1$ parts
($1$, $2 \cdot 1/2$, $4 \cdot 1/4$, etc.),
i el valor de cada part és com a màxim 1.

\subsubsection{Teoria de conjunts}

\index{teoria de conjunts}
\index{conjunt}

Un \key{conjunt} és una col·lecció d'elements.
Per exemple, el conjunt
\[X=\{2,4,7\}\]
conté els elements 2, 4 i 7.
El símbol $\emptyset$ denota un conjunt buit,
i $|S|$ denota la mida d'un conjunt $S$,
és a dir, el nombre d'elements del conjunt.
Per exemple, en el conjunt anterior, $|X|=3$.

Si un conjunt $S$ conté un element $x$,
escrivim $x \in S$,
i en cas contrari escrivim $x \notin S$.
Per exemple, en el conjunt anterior
\[4 \in X \hspace{10px}\textrm{i}\hspace{10px} 5 \notin X.\]

\begin{samepage}
Es poden construir nous conjunts mitjançant operacions de conjunts:
\begin{itemize}
\index{intersecció}
\item La \key{intersecció} $A \cap B$ \'es el conjunt dels
elements que estan tant en $A$ i en $B$.
Per exemple, si $A=\{1,2,5\}$ i $B=\{2,4\}$,
llavors $A \cap B = \{2\}$.
\index{unió}
\item La \key{union} $A \cup B$ \'es el conjunt dels
elements que estan en $A$ o en $B$ o en tots dos.
Per exemple, si $A=\{3,7\}$ i $B=\{2,3,8\}$,
llavors $A \cup B = \{2,3,7,8\}$.
\index{complement}
\index{conjunt universal}
\item El \key{complement} $\bar A$ \'es el conjunt dels
elements que no estan en $A$.
\index{conjunt universal}
La interpretació d'un complement depèn de
el \key{conjunt universal}, que \'es el conjunt que conté
tots els elements possibles.
Per exemple, si $A=\{1,2,5,7\}$ i el conjunt universal és
$\{1,2,\ldots,10\}$, aleshores $\bar A = \{3,4,6,8,9,10\}$.
\index{diferència}
\item La \key{diferència} $A \setminus B = A \cap \bar B$
\'es el conjunt dels elements que estan en $A$ però no en $B$.
Tingueu en compte que $B$ pot contenir elements que no es troben a $A$.
Per exemple, si $A=\{2,3,7,8\}$ i $B=\{3,5,8\}$,
aleshores $A \setminus B = \{2,7\}$.
\end{itemize}
\end{samepage}

\index{subconjunt}
Si cada element de $A$ també pertany a $S$,
diem que $A$ és un \key{subconjunt} de $S$,
i ho denotem $A \subset S$.
Un conjunt $S$ sempre té $2^{|S|}$ subconjunts,
inclòs el conjunt buit.
Per exemple, els subconjunts del conjunt $\{2,4,7\}$ són
\begin{center}
$\emptyset$,
$\{2\}$, $\{4\}$, $\{7\}$, $\{2,4\}$, $\{2,7\}$, $\{4,7 \}$ i $\{2,4,7\}$.
\end{center}

Alguns conjunts fets servir sovint són
$\mathbb{N}$ (nombres naturals),
$\mathbb{Z}$ (nombres enters),
$\mathbb{Q}$ (nombres racionals) i
$\mathbb{R}$ (nombres reals).
El conjunt $\mathbb{N}$
es pot definir de dues maneres, segons
la situació:
o $\mathbb{N}=\{0,1,2,\ldots\}$
o $\mathbb{N}=\{1,2,3,...\}$.

També podem construir un conjunt utilitzant una expressi\'o de la forma
\[\{f(n): n \in S\},\]
on $f(n)$ és una funció.
Aquest conjunt conté tots els elements de la forma $f(n)$,
on $n$ és un element de $S$.
Per exemple, el conjunt
\[X=\{2n : n \in \mathbb{Z}\}\]
conté tots els nombres enters parells.

\subsubsection{Lògica}

\index{lògica}
\index{negació}
\index{conjunció}
\index{disjunció}
\index{implicació}
\index{equivalència}

El valor d'una expressió lògica és
\key{cert} (1) o \key{fals} (0).
Les operacions lògiques més importants són
$\lnot$ (\key{negació}),
$\land$ (\key{conjunció}),
$\lor$ (\key{disjunció}),
$\Rightarrow$ (\key{implicació}) i
$\Leftrightarrow$ (\key{equivalència}).
La taula següent mostra el significat d'aquestes operacions:

\begin{center}
\begin{tabular}{rr|rrrrrrr}
$A$ & $B$ & $\lnot A$ & $\lnot B$ & $A \land B$ & $A \lor B$ & $A \Rightarrow B$ & $A \Leftrightarrow B$ \\
\hline
0 & 0 & 1 & 1 & 0 & 0 & 1 & 1 \\
0 & 1 & 1 & 0 & 0 & 1 & 1 & 0 \\
1 & 0 & 0 & 1 & 0 & 1 & 0 & 0 \\
1 & 1 & 0 & 0 & 1 & 1 & 1 & 1 \\
\end{tabular}
\end{center}

L'expressió $\lnot A$ té el valor oposat de $A$.
L'expressió $A \land B$ és certa si $A$ i $B$
són certes,
i l'expressió $A \lor B$ és certa si $A$ o $B$ o tots dos
són certes.
L'expressió $A \Rightarrow B$ és certa
si sempre que $A$ és cert, $B$ també ho és.
L'expressió $A \Leftrightarrow B$ és certa
quan $A$ i $B$ són les dues certes o les dues falses.

\index{predicat}

Un \key{predicat} és una expressió que és certa o falsa
en funció dels seus paràmetres.
Els predicats solen indicar-se amb majúscules.
Per exemple, considerem el predicat $P(x)$
que és cert exactament quan $x$ és un nombre primer.
Amb aquesta definició, $P(7)$ és cert però $P(8)$ és fals.

\index{quantificador}

Un \key{quantificador} connecta una expressió lògica
als elements d'un conjunt.
Els quantificadors més importants són
$\forall$ (\key{per a tots}) i $\exists$ (\key{existeix un}).
Per exemple,
\[\forall x (\exists y (y <x))\]
significa que per a cada element $x$ del conjunt,
existeix un element $y$ al conjunt
que compleix que $y$ és més petit que $x$.
Això és cert en el conjunt de nombres enters,
però fals en el conjunt dels nombres naturals.

Utilitzant la notació descrita anteriorment,
podem expressar molts tipus de proposicions lògiques.
Per exemple,
\[\forall x ((x>1 \land \lnot P(x)) \Rightarrow (\exists a (\exists b (a > 1 \land b > 1 \land x = ab))))\]
significa que si un nombre $x$ és més gran que 1
i no és un nombre primer,
aleshores hi ha els nombres $a$ i $b$
que s\'on més grans que $1$ i el producte dels quals \'es $x$.
Aquesta proposició és certa en el conjunt dels nombres enters.

\subsubsection{Funcions}

La funció $\lfloor x \rfloor$ arrodoneix (cap avall) el nombre $x$
fins a un nombre enter, i la funció
$\lceil x \rceil$ arrodoneix (cap amunt) el nombre $x$
fins a un nombre enter. Per exemple,
\[ \lfloor 3/2 \rfloor = 1 \hspace{10px} \textrm{i} \hspace{10px} \lceil 3/2 \rceil = 2.\]

Les funcions $\min(x_1,x_2,\ldots,x_n)$
i $\max(x_1,x_2,\ldots,x_n)$
donen el valor més petit i més gran dels valors
$x_1,x_2,\ldots,x_n$.
Per exemple,
\[ \min(1,2,3)=1 \hspace{10px} \textrm{i} \hspace{10px} \max(1,2,3)=3.\]

\index{factorial}

El \key{factorial} $n!$ es pot definir
\[\prod_{x=1}^n x = 1 \cdot 2 \cdot 3 \cdot \ldots \cdot n\]
o, recursivament,
\[
\begin{matrix}{lcl}
0! & = & 1 \\
n! & = & n \cdot (n-1)! \\
\end{matrix}
\]

\index{Nombres de Fibonacci}

Els \key{nombres de Fibonacci}
%\footnote{Fibonacci (c. 1175--1250) va ser un matemàtic italià.}
sorgeixen en moltes situacions.
Es poden definir recursivament de la següent manera:
\[
\begin{matrix}{lcl}
f(0) & = & 0 \\
f(1) & = & 1 \\
f(n) & = & f(n-1)+f(n-2) \\
\end{matrix}
\]
Els primers nombres de Fibonacci són
\[0, 1, 1, 2, 3, 5, 8, 13, 21, 34, 55, \ldots\]
També hi ha una fórmula tancada
per calcular els nombres de Fibonacci, que de vegades s'anomena
\index{Fórmula de Binet} \key{fórmula de Binet}:
\[f(n)=\frac{(1 + \sqrt{5})^n - (1-\sqrt{5})^n}{2^n \sqrt{5}}.\]

\subsubsection{Logaritmes}

\index{logaritme}

El \key{logaritme} d'un nombre $x$
es denota per $\log_k(x)$, on $k$ és la base
del logaritme.
Segons la definició,
$\log_k(x)=a$ exactament quan $k^a=x$.

Una propietat útil dels logaritmes és
que $\log_k(x)$ és igual al nombre de vegades
hem de dividir $x$ per $k$ abans d'arribar
el nombre 1.
Per exemple, $\log_2(32)=5$
perquè calen 5 divisions per 2:

\[32 \rightarrow 16 \rightarrow 8 \rightarrow 4 \rightarrow 2 \rightarrow 1 \]

Els logaritmes s'utilitzen sovint en l'anàlisi de
algorismes, perquè molts algorismes eficients
redueixen alguna cosa a la meitat a cada pas.
Per tant, podem estimar l'eficiència d'aquests algorismes
fent servir logaritmes.

El logaritme d'un producte és
\[\log_k(ab) = \log_k(a)+\log_k(b),\]
i en conseqüència,
\[\log_k(x^n) = n \cdot \log_k(x).\]
A més, el logaritme d'un quocient és
\[\log_k\Big(\frac{a}{b}\Big) = \log_k(a)-\log_k(b).\]
Una altra fórmula útil és
\[\log_u(x) = \frac{\log_k(x)}{\log_k(u)},\]
i amb això, és possible calcular
logaritmes a qualsevol base si hi ha una manera de
calcular logaritmes en una base fixa.

\index{logaritme natural}

El \key{logaritme natural} $\ln(x)$ d'un nombre $x$
és el logaritme la base del qual és $e \approx 2.71828$.
Una altra propietat dels logaritmes és que
el nombre de dígits d'un enter $x$ en base $b$ és
$\lfloor \log_b(x)+1 \rfloor$.
Per exemple, la representació de
$ 123 $ en base $ 2 $ és 1111011 i
$\lfloor \log_2(123)+1 \rfloor = 7$.

\section{Concursos i recursos}

\subsubsection{IOI}

L'Olimpíada Internacional d'Informàtica (IOI,
International Olympiad in Informatics)
és un concurs de programació anual per a
alumnes de secundària.
Cada país pot enviar un equip de
quatre alumnes al concurs.
Normalment hi ha uns 300 participants
de 80 països.

L'IOI consta de dos concursos de cinc hores de durada.
En ambdós concursos, es demana als participants
resoldre tres tasques algor\'ismiques de diferent dificultat.
Les tasques es divideixen en subtasques,
cadascuna dels quals té una puntuació assignada.
Encara que els concursants estiguin dividits en equips,
competeixen com a individus.

El pla d'estudis de l'IOI \cite{iois} regula els temes
que poden aparèixer a les tasques IOI.
Gairebé tots els temes del temari de l'IOI
són coberts per aquest llibre.

Els participants de l'IOI es seleccionen mitjançant
concursos nacionals.
Abans de l'IOI, s'organitzen molts concursos regionals,
com l'Olimpíada Bàltica d'Informàtica (BOI),
Olimpíada Centro-Europea d'Informàtica (CEOI)
i l'Olimpíada d'Informàtica Àsia-Pacífic (APIO).

Alguns països organitzen concursos de pràctiques on-line
per als futurs participants de l'IOI,
com el Concurs Obert d'Informàtica de Croàcia \cite{coci}
i l'Olimpíada d'Informàtica dels EUA \cite{usaco}.
A més, hi ha una gran col·lecció de problemes dels concursos polonesos
disponibles on-line línia \cite{main}.

\subsubsection{ICPC}

El Concurs Internacional de Programació Col·legial (ICPC,
International Collegiate Programming Contest)
és un concurs de programació anual per a estudiants universitaris.
Cada equip del concurs està format per tres estudiants,
i a diferència de l'IOI, els alumnes treballen conjuntament;
només hi ha un ordinador disponible per a cada equip.

L'ICPC consta de diverses fases, i nom\'es els millors
equips estan convidats a la fase final (World Finals).
Tot i que hi ha desenes de milers de participants
al concurs, només hi ha un petit nombre\footnote{El nombre exacte d'equips
a la fase final varia d'un any a un altre; el 2017, hi havia
133 equips.} d'espais per equips, de manera que fins i tot avançar a
la final ja és un gran èxit en algunes regions.

En cada concurs de l'ICPC, els equips disposen de cinc hores de temps
per a resoldre uns deu problemes algor\'ismics.
La solució d'un problema només s'accepta si es resolen
tots els casos de prova de manera eficient.
Durant el concurs, els competidors poden veure els resultats d'altres equips,
però durant l'última hora el marcador està congelat i no \'es
possible veure els resultats dels últims enviaments.

Els temes que poden aparèixer a l'ICPC no estan tan bé
especificats com els de l'IOI.
En tot cas, és evident que calen més coneixements
a l'ICPC, sobretot més habilitats matemàtiques.

\subsubsection{Concursos on-line}

També hi ha molts concursos on-line oberts a tothom.
En l'actualitat, la p\`agina web amb més concursos actius és Codeforces,
que organitza concursos de forma setmanal.
A Codeforces, els participants es divideixen en dues divisions:
els principiants competeixen a la Div2 i els programadors més experimentats a
la Div1.
Altres llocs de concurs inclouen AtCoder, CS Academy, HackerRank i Topcoder.

Algunes empreses organitzen concursos on-line amb finals presencials.
Exemples d'aquests concursos són Facebook Hacker Cup,
Google Code Jam i Yandex.Algorithm.
Per descomptat, les empreses també utilitzen aquests concursos per a reclutar
programadors:
tenir un bon resultat en un concurs és una bona manera de demostrar les teves habilitats.

\subsubsection{Llibres}

Existeixen alguns llibres (a més d'aquest llibre) que
es centren en la programació competitiva i la resolució de problemes
algorísmics:

\begin{itemize}
\item S. S. Skiena i M. A. Revilla:
\emph{Programming Challenges: The Programming Contest Training Manual} \cite{ski03}
\item S. Halim i F. Halim:
\emph{Competitive Programming 3: The New Lower Bound of Programming Contests} \cite{hal13}
\item K. Diks et al.: \emph{Looking for a Challenge? The Ultimate Problem Set from
the University of Warsaw Programming Competitions} \cite{dik12}
\end{itemize}

Els dos primers llibres estan pensats per a principiants,
mentre que l'últim llibre conté material avançat.

Per descomptat, els llibres generals d'algorismia també són adequats
per als programadors competitius.
Alguns llibres populars són:

\begin{itemize}
\item T. H. Cormen, C. E. Leiserson, R. L. Rivest i C. Stein:
\emph{Introduction to Algorithms} \cite{cor09}
\item J. Kleinberg i É. Tardes:
\emph{Algorithm Design} \cite{kle05}
\item S. S. Skiena:
\emph{The Algorithm Design Manual} \cite{ski08}
\end{itemize}
