\chapter*{Prefaci}
\markboth{\MakeUppercase{Prefaci}}{}
\addcontentsline{toc}{chapter}{Prefaci}
L'objectiu d'aquest llibre és donar-vos una introducció completa a la
programació competitiva. Se suposa que ja coneixeu els fonaments
bàsics de la programació, però no cal cap formació prèvia en
programació competitiva.

El llibre està especialment pensat per a estudiants que vulguin
aprendre algorismes i possiblement participar en l'Olimpíada
Internacional d'Informàtica (IOI) o en el Concurs Internacional de
Programació Col·legiata (ICPC). Per descomptat, el llibre també és
adequat per a qualsevol altra persona interessada en la programació
competitiva.

Es necessita molt de temps per convertir-se en un bon programador
competitiu, però també és una oportunitat per aprendre molt. Podeu
estar segur que obtindreu una bona comprensió general dels algorismes
si passeu temps llegint el llibre, resolent problemes i participant en
concursos.

El llibre està en desenvolupament continu. Sempre podeu enviar
comentaris sobre el llibre a \texttt{ahslaaks@cs.helsinki.fi}.

\begin{flushright}
Helsinki, August 2019 \\
Antti Laaksonen
\end{flushright}

\bigskip
\bigskip

(Nota del Traductor) L'Antti Laaksonen és l'autor de ``Guide to
Competitive Programming'' (Springer, 2017). Aquest PDF és la traducció
al català del seu manual ``Competitive Programmer's Handbook''
(\texttt{https://github.com/pllk/cphb}). Aquesta traducció no hagués
estat possible si l'Antti no hagués distribuit les fonts \LaTeX\ del
seu manual sota llicència Creative Common.

Amb aquesta traducció vull permetre que els participants de
l'Olimpíada Informàtica de Catalunya \emph{que no saben prou d'anglès}
puguin millorar el seu nivell de programació competitiva. Si ets una
d'aquestes persones, espero que aquesta traducció et sigui útil! Però
recorda que no poder llegir en anglès és literalment una
\emph{minusvalia} que t'obliga a dependre de la voluntat i criteri de
traductors com jo. Si vols continuar aprenent coses, millora el teu
anglès.

\begin{flushright}
Eugene, Desembre 2021 \\
Omer Giménez Llach
\end{flushright}

