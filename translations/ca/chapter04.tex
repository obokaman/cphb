\chapter{Estructures de dades}

\index{estructura de dades}

Una \key{estructura de dades} és una manera d'emmagatzemar
dades a la memòria d'un ordinador.
És important escollir l'estructura
de dades adequada per a un problema,
perquè cada estructura de dades té els seus
avantatges i inconvenients.
La pregunta crucial és: quines operacions
són eficients en l'estructura de dades escollida?

Aquest capítol presenta les estructures de dades
més importants a la biblioteca estàndard de C++.
És molt bona idea fer servir la biblioteca estàndard
sempre que sigui possible,
perquè ens estalviarà molt de temps.
Més endavant parlarem d'estructures de dades més sofisticades
no estan disponibles a la biblioteca estàndard.

\section{Vectors dinàmics}

\index{vector dinàmic}
\index{vector}
\index{array}

Un \key{vector dinàmic} és un vector la mida
del qual pot canviar durant l'execució
del programa.
El vector dinàmic més popular en C++ és
l'estructura \texttt{vector},
que també pot fer-se servir gairebé com un array de C normal.

El codi següent crea un vector buit i
hi afegeix tres elements:

\begin{lstlisting}
vector<int> v;
v.push_back(3); // [3]
v.push_back(2); // [3,2]
v.push_back(5); // [3,2,5]
\end{lstlisting}

Després d'això, podem accedir als elements com si fos un array de C
normal:

\begin{lstlisting}
cout << v[0] << "\n"; // 3
cout << v[1] << "\n"; // 2
cout << v[2] << "\n"; // 5
\end{lstlisting}

La funció \texttt{size} retorna el nombre d'elements del vector.
El codi següent itera
el vector i escriu tots els elements:

\begin{lstlisting}
for (int i = 0; i < v.size(); i++) {
    cout << v[i] << "\n";
}
\end{lstlisting}

Una manera més concisa d'iterar un vector és la següent:

\begin{lstlisting}
for (auto x : v) {
    cout << x << "\n";
}
\end{lstlisting}

La funció \texttt{back} retorna l'últim element
en el vector, i la funció \texttt{pop\_back} elimina l'últim element:

\begin{lstlisting}
vector<int> v;
v.push_back(5);
v.push_back(2);
cout << v.back() << "\n"; // 2
v.pop_back();
cout << v.back() << "\n"; // 5
\end{lstlisting}

El codi següent crea un vector de cinc elements:

\begin{lstlisting}
vector<int> v = {2,4,2,5,1};
\end{lstlisting}

Una altra manera de crear un vector és donar el nombre
d'elements i el valor inicial de cada element:

\begin{lstlisting}
// mida 10, valor inicial 0
vector<int> v(10);
\end{lstlisting}
\begin{lstlisting}
// mida 10, valor inicial 5
vector<int> v(10, 5);
\end{lstlisting}

La implementació interna d'un vector
fa servir un array ordinari.
Si la mida del vector augmenta i
l'espai reservat es massa petit,
es crea un nou array i tot
els elements es mouen al nou espai
Però això no passa sovint, i la
complexitat temporal promig de fer un
\texttt{push\_back} és $O(1)$.

\index{string}
\index{cadena}

L'estructura cadena (\texttt{string})
és un vector dinàmic de caràcters.
A més, hi ha una sintaxi especial per a les cadenes
que no està disponible en les altres estructures de dades.
Les cadenes es poden concatenar fent servir el símbol \texttt{+}.
La funció $\texttt{substr}(k,x)$ retorna la subcadena
que comença a la posició $k$ i té longitud $x$,
i la funció $\texttt{find}(\texttt{t})$ troba la posició
de la primera ocurrència d'una subcadena \texttt{t}.

El codi següent presenta algunes operacions de cadenes:

\begin{lstlisting}
string a = "hatti";
string b = a+a;
cout << b << "\n"; // hattihatti
b[5] = 'v';
cout << b << "\n"; // hattivatti
string c = b.substr(3,4);
cout << c << "\n"; // tiva
\end{lstlisting}

\section{Estructures conjunt}

\index{set}
\index{unordered_set}
\index{conjunt}

Un \key{conjunt} és una estructura de dades que
manté una col·lecció d'elements.
Les operacions bàsiques dels conjunts són
inserir, cercar i eliminar.

La biblioteca estàndard de C++ conté dues implementacions:
L'estructura \texttt{set} es basa en arbres binari
equilibrats i les seves operacions triguen $O(\log n)$ en cas pitjor.
L'estructura \texttt{unordered\_set} fa serving hashing,
i les seves operacions triguen $O(1)$ en mitjana.

Quina implementació triar és
sovint qüestió de gustos.
El benefici de l'estructura \texttt{set}
és que manté l'ordre dels elements
i ofereix funcions que no estan disponibles
a \texttt{unordered\_set}.
D'altra banda, \texttt{unordered\_set}
pot ser més eficient.

El codi següent crea un conjunt
que conté nombres enters,
i mostra algunes de les operacions.
La funció \texttt{insert} afegeix un element al conjunt,
la funció \texttt{count} retorna el nombre d'ocurrències
d'un element del conjunt,
i la funció \texttt{erase} elimina un element del conjunt.

\begin{lstlisting}
set<int> s;
s.insert(3);
s.insert(2);
s.insert(5);
cout << s.count(3) << "\n"; // 1
cout << s.count(4) << "\n"; // 0
s.erase(3);
s.insert(4);
cout << s.count(3) << "\n"; // 0
cout << s.count(4) << "\n"; // 1
\end{lstlisting}

Un conjunt es pot fer servir com un vector,
però no es pot accedir
als elements fent servir la notació \texttt{[]}.
El codi següent crea un conjunt,
escriu el nombre d'elements, i
itera per tots els elements:
\begin{lstlisting}
set<int> s = {2,5,6,8};
cout << s.size() << "\n"; // 4
for (auto x : s) {
    cout << x << "\n";
}
\end{lstlisting}

Una propietat important dels conjunts és
que tots els seus elements són \emph{diferents}.
Així, la funció \texttt{count} sempre retorna
o 0 (l'element no està al conjunt)
o 1 (l'element és al conjunt),
i la funció \texttt{insert} no s'afegeix mai
un element al conjunt si ja hi és.
Això es pot veure al codi següent:

\begin{lstlisting}
set<int> s;
s.insert(5);
s.insert(5);
s.insert(5);
cout << s.count(5) << "\n"; // 1
\end{lstlisting}

C++ també conté les estructures
\texttt{multiset} i \texttt{unordered\_multiset},
que funcionen com \texttt{set}
i \texttt{unordered\_set}
excepte que poden contenir vàries instàncies d'un element.
Per exemple, al codi següent afegeix tres instàncies
del número 5 a un multiconjunt:

\begin{lstlisting}
multiset<int> s;
s.insert(5);
s.insert(5);
s.insert(5);
cout << s.count(5) << "\n"; // 3
\end{lstlisting}
La funció \texttt{erase} elimina
totes les instàncies d'un element
d'un multiconjunt:
\begin{lstlisting}
s.erase(5);
cout << s.count(5) << "\n"; // 0
\end{lstlisting}
Sovint només volem eliminar una instància,
que podem fer de la següent manera:
\begin{lstlisting}
s.erase(s.find(5));
cout << s.count(5) << "\n"; // 2
\end{lstlisting}

\section{Estructures mapa}

\index{mapa}
\index{map}
\index{unordered_map}

Un \key{mapa} és un vector que consisteix
de parells clau-valor.
Mentre que les claus d'un vector normal sempre
són els nombres enters consecutius $0,1,\ldots,n-1$,
on $n$ és la mida del vector,
les claus d'un mapa poden ser de qualsevol tipus de dades i
no tenen perquè ser consecutius.

La biblioteca estàndard de C++ conté dos implementacions
de mapa que es corresponen amb les implmentacions de
conjunts: l'estructura
\texttt{map} es basa en un arbre binari
equilibrat i accedir als elements triga $O(\log n)$,
mentre que l'estructura
\texttt{unordered\_map} fa servir hashing
i accedir als elements triga $O(1)$ de mitjana.

El codi següent crea un mapa
on les claus són cadenes i els valors són nombres enters:

\begin{lstlisting}
map<string,int> m;
m["monkey"] = 4;
m["banana"] = 3;
m["harpsichord"] = 9;
cout << m["banana"] << "\n"; // 3
\end{lstlisting}

Si es demana el valor d'una clau
però el mapa no la conté,
la clau s'afegeix automàticament al mapa amb
un valor per defecte.
Per exemple, en el codi següent,
la clau ``aybabtu'' s'afegeix al mapa
amb valor 0.

\begin{lstlisting}
map<string,int> m;
cout << m["aybabtu"] << "\n"; // 0
\end{lstlisting}
La funció \texttt{count} comprova
si el mapa conté una clau:
\begin{lstlisting}
if (m.count("aybabtu")) {
    // la clau existeix
}
\end{lstlisting}
El codi següent escriu totes les claus i valors
d'un mapa:
\begin{lstlisting}
for (auto x : m) {
    cout << x.first << " " << x.second << "\n";
}
\end{lstlisting}

\section{Iteradors i intervals}

\index{iterador}

Moltes funcions de la biblioteca estàndard de C++
operen amb iteradors.
Un \key{iterador} (\texttt{iterator}) és una variable que
apunta a un element d'una estructura de dades.

Els iteradors \texttt{begin}
i \texttt{end} defineixen un interval que conté
tots els elements d'una estructura de dades.
L'iterador \texttt{begin} apunta
al primer element de l'estructura de dades,
i l'iterador \texttt{end} apunta
a la posició \emph{després de} l'últim element.
Per exemple:

\begin{center}
\begin{tabular}{llllllllll}
\{ & 3, & 4, & 6, & 8, & 12, & 13, & 14, & 17 & \} \\
& $\uparrow$ & & & & & & & & $\uparrow$ \\
& \multicolumn{3}{l}{\texttt{s.begin()}} & & & & & & \texttt{s.end()} \\
\end{tabular}
\end{center}

Observeu l'asimetria dels iteradors:
\texttt{s.begin()} apunta a un element de l'estructura de dades,
mentre que \texttt{s.end()} apunta fora de l'estructura de dades.
És a dir, l'interval definit pels iteradors és \emph{semi-obert}.

\subsubsection{Treballar amb intervals}

Els iteradors es fan servir en la biblioteca estàndard de C++
per a passar intervals d'elements en una estructura de dades.
Normalment, volem processar tots els elements d'una
estructura de dades, i per tant fem servir els iteradors
\texttt{begin} i \texttt{end}.

Per exemple, el codi següent ordena un vector
fent servir la funció \texttt{sort},
després inverteix l'ordre dels elements fent servir la funció
\texttt{reverse}, i finalment barreja l'ordre de
els elements fent servir la funció \texttt{random\_shuffle}.

\index{sort@\texttt{sort}}
\index{reverse@\texttt{reverse}}
\index{random\_shuffle@\texttt{random\_shuffle}}

\begin{lstlisting}
sort(v.begin(), v.end());
reverse(v.begin(), v.end());
random_shuffle(v.begin(), v.end());
\end{lstlisting}

Aquestes funcions també es poden fer servir
amb arrays ordinaris de C. En aquest cas, les funcions reben
punters a l'array en lloc d'iteradors:

\begin{lstlisting}
sort(a, a+n);
reverse(a, a+n);
random_shuffle(a, a+n);
\end{lstlisting}

\subsubsection{Iteradors de conjunts}

Els iteradors es fan servir sovint per a accedir als
elements d'un conjunt.
El codi següent crea un iterador
\texttt{it} que assenyala a l'element més petit d'un conjunt:

\begin{lstlisting}
set<int>::iterator it = s.begin();
\end{lstlisting}
També es pot escriure així:
\begin{lstlisting}
auto it = s.begin();
\end{lstlisting}
Per a accedir a l'element que l'iterador assenyala es
fa servir el símbol \texttt{*}.
Per exemple, el següent codi imprimeix
el primer element del conjunt:

\begin{lstlisting}
auto it = s.begin();
cout << *it << "\n";
\end{lstlisting}

Els iteradors es poden moure mitjançant els operadors
\texttt{++} (endavant) i \texttt{--} (enrere),
és a dir, moure l'iterador a l'element següent
o l'element anterior.

El codi següent escriu tots els elements
en ordre creixent:
\begin{lstlisting}
for (auto it = s.begin(); it != s.end(); it++) {
    cout << *it << "\n";
}
\end{lstlisting}
El codi següent escriu l'element més gran del conjunt:
\begin{lstlisting}
auto it = s.end(); it--;
cout << *it << "\n";
\end{lstlisting}

La funció $\texttt{find}(x)$ retorna un iterador
que apunta a un element el valor del qual és $x$.
Si el conjunt no conté $x$,
l'iterador retornat serà \texttt{end}.

\begin{lstlisting}
auto it = s.find(x);
if (it == s.end()) {
    // no trobem x
}
\end{lstlisting}

La funció $\texttt{lower\_bound}(x)$ retorna
un iterador al primer element del conjunt
el valor del qual és \emph{almenys} $x$, i
la funció $\texttt{upper\_bound}(x)$
retorna un iterador al primer element del conjunt
el valor del qual és \emph{més gran que} $x$.
En ambdos casos, si aquest primer element del conjunt
no existeix, retornen \texttt{end}.\footnote{N. del T.: És a dir,
$[\texttt{lower\_bound}(x), \texttt{upper\_bound}(x))$ és l'interval
semi-obert que assenyala als elements amb valor $x$.}
Aquestes funcions no són compatibles amb
l'estructura \texttt{unordered\_set} perquè aquesta
no manté l'ordre dels elements.

Per exemple, el codi següent troba l'element
més proper a $x$:

\begin{lstlisting}
auto it = s.lower_bound(x);
if (it == s.begin()) {
    cout << *it << "\n";
} else if (it == s.end()) {
    it--;
    cout << *it << "\n";
} else {
    int a = *it; it--;
    int b = *it;
    if (x-b < a-x) cout << b << "\n";
    else cout << a << "\n";
}
\end{lstlisting}

El codi suposa que el conjunt no està buit,
i repassa tots els casos possibles
utilitzant un iterador \texttt{it}.
L'iterador apunta al valor més petit
el valor del qual és almenys $x$.
Si \texttt{it} és igual a \texttt{begin},
l'element corresponent és el més proper a $x$.
Si \texttt{it} és igual a \texttt{end},
l'element més gran del conjunt és el més proper a $x$.
Si no es compleix cap dels dos casos anteriors,
l'element més proper a $x$ és o bé l'element assenyalat
per \texttt{it} o bé l'element anterior.

\section{Altres estructures}

\subsubsection{Conjunt de bits}

\index{set de bits}

Un conjunt de bits, o \key{bitset}, és un vector
on cada valor és 0 o 1.
El codi següent crea un conjunt de bits amb 10 elements:
\begin{lstlisting}
bitset<10> s;
s[1] = 1;
s[3] = 1;
s[4] = 1;
s[7] = 1;
cout << s[4] << "\n"; // 1
cout << s[5] << "\n"; // 0
\end{lstlisting}

L'avantatge de fer servir conjunts de bits és
que requereixen menys memòria que els vectors normals,
perquè cada element només fa servir un únic bit de memòria.
Per exemple,
si emmagatzemèssim $n$ bits en un vector de tipus \texttt{int},
necessitaríem $32n$ bits de memòria,
però en un conjunt de bits només fan falta $n$ bits de memòria.
A més, els valors d'un conjunt de bits
es poden manipular de manera eficient fent servir
operadors de bits, amb la qual cosa encara és possible
optimitzar més alguns algorismes.

El codi següent mostra una altra manera de crear el conjunt de bits
anterior:
\begin{lstlisting}
bitset<10> s(string("0010011010")); // de dreta a esquerra
cout << s[4] << "\n"; // 1
cout << s[5] << "\n"; // 0
\end{lstlisting}

La funció \texttt{count} retorna el nombre
d'uns en el conjunt de bits:

\begin{lstlisting}
bitset<10> s(string("0010011010"));
cout << s.count() << "\n"; // 4
\end{lstlisting}

El codi següent mostra exemples d'operacions de bits:
\begin{lstlisting}
bitset<10> a(string("0010110110"));
bitset<10> b(string("1011011000"));
cout << (a&b) << "\n"; // 0010010000
cout << (a|b) << "\n"; // 1011111110
cout << (a^b) << "\n"; // 1001101110
\end{lstlisting}

\subsubsection{Deque}

\index{deque}

Una \key{deque} és un vector dinàmic
la mida del qual es pot canviar eficientment
a ambdós extrems del vector.
Al igual que els vectors, un deque proporciona les funcions
\texttt{push\_back} i \texttt{pop\_back}, però
també inclou les funcions
\texttt{push\_front} i \texttt{pop\_front}
que no estan disponibles en un vector normal.

Un deque es pot fer servir de la manera següent:
\begin{lstlisting}
deque<int> d;
d.push_back(5); // [5]
d.push_back(2); // [5,2]
d.push_front(3); // [3,5,2]
d.pop_back(); // [3,5]
d.pop_front(); // [5]
\end{lstlisting}

La implementació interna d'un deque
és més complexa que el d'un vector,
i per aquest motiu, un deque és més lent que un vector.
Tot i així, afegir o eliminar
elements per qualsevol dels dos extrems triga $O(1)$ de mitjana.

\subsubsection{Pila}

\index{pila}

Una \key{pila} (\texttt{stack})
és una estructura de dades que proporciona dos
operacions que triguen $O(1)$:
afegint un element a la part superior,
i retirar l'element de la part superior.

El codi següent mostra com fer servir una pila:
\begin{lstlisting}
stack<int> s;
s.push(3);
s.push(2);
s.push(5);
cout << s.top(); // 5
s.pop();
cout << s.top(); // 2
\end{lstlisting}

\subsubsection{Queue}

\index{queue}

Una \key{queue} també
ofereix dos operacions de temps $O(1)$:
afegir un element al final de la cua,
i treure el primer element de la cua.
Només és possible accedir al primer i al darrer element de la cua.

El codi següent mostra com fer servir una cua:
\begin{lstlisting}
queue<int> q;
q.push(3);
q.push(2);
q.push(5);
cout << q.front(); // 3
q.pop();
cout << q.front(); // 2
\end{lstlisting}

\subsubsection{Cua de prioritats}

\index{cua de prioritats}
\index{heap}

Una \key{cua de prioritats}
manté un conjunt d'elements.
S'ofereixen les operacions d'inserció i,
depenent del tipus de cua,
recuperar o eliminar
l'element mínim o màxim (només un dels dos).
Inserir i eliminar elements triga $O(\log n)$,
mentre que obtenir l'element mínim o màxim triga $O(1)$.

En principi, es podria fer servir un conjunt ordenat per
a implementar una cua de prioritats, però l'avantatge de fer
servir una cua de prioritats és que els factors constants són més
petits.
Les cues de prioritats s'implementen fent servir una estructura
de \texttt{heap} que és molt més senzilla que els arbres binaris
balancejats dels conjunts ordenats.

Per defecte, els elements en una cua de prioritats
de C++ s'ordenen en ordre decreixent,
i només podem trobar i eliminar l'element més gran de la cua.
Per exemple:

\begin{lstlisting}
priority_queue<int> q;
q.push(3);
q.push(5);
q.push(7);
q.push(2);
cout << q.top() << "\n"; // 7
q.pop();
cout << q.top() << "\n"; // 5
q.pop();
q.push(6);
cout << q.top() << "\n"; // 6
q.pop();
\end{lstlisting}

Si volem crear una cua de prioritat
que permeti trobar i eliminar
l'element més petit,
podem fer el següent:

\begin{lstlisting}
priority_queue<int,vector<int>,greater<int>> q;
\end{lstlisting}

\subsubsection{Estructures de dades ``policy-based''}

El compilador \texttt{g++} també ofereix algunes
estructures de dades que no formen part
de la biblioteca estàndard C++.
Aquestes estructures s'anomenen \emph{policy-based}.
Per a fer-les servir, hem d'afegir les línies següents:
\begin{lstlisting}
#include <ext/pb_ds/assoc_container.hpp>
using namespace __gnu_pbds;
\end{lstlisting}
Després d'això, podem fer servir una estructura de dades
\texttt{indexed\_set} que
és un \texttt{set} que es pot indexar com un vector.
La definició per a valors de tipus \texttt{int} és la següent:
\begin{lstlisting}
typedef tree<int,null_type,less<int>,rb_tree_tag,
             tree_order_statistics_node_update> indexed_set; 
\end{lstlisting}
Ara podem crear un conjunt de la manera següent:
\begin{lstlisting}
indexed_set s;
s.insert(2);
s.insert(3);
s.insert(7);
s.insert(9);
\end{lstlisting}
L'especialitat d'aquest conjunt és que tenim accés
els índexs que tindrien els elements en un vector ordenat.
La funció $\texttt{find\_by\_order}$ retorna
un iterador a l'element que ocupa una posició determinada:
\begin{lstlisting}
auto x = s.find_by_order(2);
cout << *x << "\n"; // 7
\end{lstlisting}
I la funció $\texttt{order\_of\_key}$
retorna la posició d'un element donat:
\begin{lstlisting}
cout << s.order_of_key(7) << "\n"; // 2
\end{lstlisting}
Si l'element no apareix al conjunt,
obtenim la posició que tindria l'element
al conjunt:
\begin{lstlisting}
cout << s.order_of_key(6) << "\n"; // 2
cout << s.order_of_key(8) << "\n"; // 3
\end{lstlisting}
Ambdues funcions funcionen en temps logarítmic.

\section{Comparació amb l'ordenació}

Sovint és possible resoldre un problema
fent servir estructures de dades o ordenació.
De vegades hi ha diferències notables
en l'eficiència real d'aquests enfocaments,
que poden quedar amagats en les seves complexitats temporals.

Considerem un problema on
se'ns donen dues llistes $A$ i $B$, ambdues de
$n$ elements.
La nostra tasca és calcular el nombre d'elements
que pertanyen a les dues llistes.
Per exemple, per a les llistes
\[A = [5,2,8,9] \hspace{10px} \textrm{i} \hspace{10px} B = [3,2,9,5],\]
la resposta és 3 perquè els números 2, 5
i 9 pertanyen a les dues llistes.

La solució directa al problema és
recórrer tots els parells d'elements en temps $O(n^2)$,
però a continuació ens centrarem
en els algorismes més eficients.

\subsubsection{Algorisme 1}

Construïm un conjunt amb els elements que apareixen a $A$,
i després d'això, iterem a través dels elements
de $B$ i comprovem si cadascun dels elements també
pertanyen a $A$.
Això és eficient perquè els elements de $A$
estan en un conjunt.
Utilitzant l'estructura \texttt{set},
la complexitat temporal de l'algorisme és $O(n \log n)$.

\subsubsection{Algorisme 2}

No necessitem mantenir un conjunt ordenat per a $A$,
per tant, en lloc d'un \texttt{set}
fem servir un \texttt{unordered\_set}.
Aquesta és una manera fàcil de fer l'algorisme
més eficient, perquè només hem de canviar
l'estructura de dades, sense canviar l'algorisme.
La complexitat temporal del nou algorisme és $O(n)$.

\subsubsection{Algorisme 3}

Ordenem en lloc de fer servir estructures de dades.
Primer, ordenem les dues llistes $A$ i $B$.
Després d'això, recorrem les dues llistes
alhora i trobem els elements comuns.
El cost de l'ordenació és $O(n \log n)$,
i la resta de l'algorisme triga $O(n)$,
per tant, la complexitat total és $O(n \log n)$.

\subsubsection{Comparació d'eficiència}

La taula següent mostra l'eficiència
els algorismes anteriors quan $n$ varia i
els elements de les llistes són nombres enters
aleatoris entre $1 \ldots 10^9$:

\begin{center}
\begin{tabular}{rrrr} 
$n$ & Algorisme 1 & Algorisme 2 & Algorisme 3 \\
\hline
$10^6$ & $1,5$ s & $0,3$ s & $0,2$ s \\
$2 \cdot 10^6$ & $3,7$ s & $0,8$ s & $0,3$ s \\
$3 \cdot 10^6$ & $5,7$ s & $1,3$ s & $0,5$ s \\
$4 \cdot 10^6$ & $7,7$ s & $1,7$ s & $0,7$ s \\
$5 \cdot 10^6$ & $10.0$ s & $2.3$ s & $0.9$ s \\
\end{tabular}
\end{center}

Els Algorismes 1 i 2 són iguals excepte que fem servir
diferents estructures de conjunt.
En aquest problema, aquesta elecció té un efecte important
en el temps d'execució, perquè l'Algorisme 2
és de 4 a 5 vegades més ràpid que l'Algorisme 1.

Tanmateix, l'algorisme més eficient és el que fa servir l'ordenació,
l'Algorisme 3.
Només triga la meitat del temps que triga l'Algorisme 2.
Curiosament, la complexitat temporal tant de
l'Algorisme 1 com de l'Algorisme 3 és $O(n \log n)$
però, malgrat això, l'Algorisme 3 és deu vegades més ràpid.
Això es pot explicar pel fet que
ordenar és un procediment senzill i només s'executa
una vegada al començament de l'Algorisme 3,
i la resta de l'algorisme triga temps lineal.
Per altra banda,
l'Algorisme 1 manté un arbre binari equilibrat complex
durant tot l'algorisme.\footnote{N. del T.: L'Algorisme 2 triga
$O(n)$, però també és més lent a la pràctica que
l'Algorisme 3, que triga temps $O(n\log n)$. En
principi, valors cada cops més grans de $n$ haurien d'afavorir
l'Algorisme 2, fins a ser un nombre arbitràriament gran de vegades
més ràpid que l'Algorisme 3. A la pràctica, $O(\log n)$ creix molt
lentament; a més a més, algunes operacions, com ara fer servir memòria
addicional, poden ser més costoses com més gran sigui la $n$
i això pot negar l'avantatge teòric de ser $O(\log n)$ cops més ràpid.}
